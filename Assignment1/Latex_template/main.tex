\documentclass[a4paper]{article}

%% Language and font encodings
\usepackage[english]{babel}
\usepackage[utf8x]{inputenc}
\usepackage[T1]{fontenc}
\usepackage{listings}

%% Sets page size and margins
\usepackage[a4paper,top=3cm,bottom=2cm,left=3cm,right=3cm,marginparwidth=1.75cm]{geometry}

%% Useful packages
\usepackage{amsmath}
\usepackage[colorinlistoftodos]{todonotes}
\usepackage[colorlinks=true, allcolors=blue]{hyperref}
\usepackage{listings}
\usepackage{url}
\usepackage{graphicx}
\graphicspath{ {./images/} }
% \DeclareGraphicsExtensions{.pdf,.jpg,.png}

%% defined colors
\definecolor{Blue}{rgb}{0,0,0.5}
\definecolor{Green}{rgb}{0,0.75,0.0}
\definecolor{LightGray}{rgb}{0.6,0.6,0.6}
\definecolor{DarkGray}{rgb}{0.3,0.3,0.3}

% \newcommand*\lstinputpath[1]{\lstset{inputpath=#1}}
% \lstinputpath{code/}

\lstset{language=R,
   basicstyle=\ttfamily\small,
   breaklines=true,
   keywordstyle=\bfseries\color{Blue},
   commentstyle=\itshape\color{LightGray},
   stringstyle=\color{Green},
   numbers=left,
   numberstyle=\tiny\color{DarkGray},
   stepnumber=1,
   numbersep=10pt,
   backgroundcolor=\color{white},
   tabsize=2,
   showspaces=false,
   showstringspaces=false,
   captionpos=b,
%    inputpath={code/},
   frame=tb
}

\lstset{language=Python,
   basicstyle=\ttfamily\small,
   breaklines=true,
   keywordstyle=\bfseries\color{Blue},
   commentstyle=\itshape\color{LightGray},
   stringstyle=\color{Green},
   numbers=left,
   numberstyle=\tiny\color{DarkGray},
   stepnumber=1,
   numbersep=10pt,
   backgroundcolor=\color{white},
   tabsize=2,
   showspaces=false,
   showstringspaces=false,
   captionpos=b,
%    inputpath={code/},
   frame=tb
}

\lstset{language=Matlab,
   basicstyle=\ttfamily\small,
   breaklines=true,
   keywordstyle=\bfseries\color{Blue},
   commentstyle=\itshape\color{LightGray},
   stringstyle=\color{Green},
   numbers=left,
   numberstyle=\tiny\color{DarkGray},
   stepnumber=1,
   numbersep=10pt,
   backgroundcolor=\color{white},
   tabsize=2,
   showspaces=false,
   showstringspaces=false,
   captionpos=b,
%    inputpath={code/},
   frame=tb
}

\title{Assignment 0: IDS latex template}
\author{
Name1 (s1234567) \\ 
Name2 (s7654321) \\ 
 \\ \textbf{Group} 1234}
\date{\today}


\begin{document}
\maketitle

\todo{Group is combined with author which is somewhat clumsy. }
\todo{Let me know what you think.}

\section*{Latex template structure}

Code and images are placed in the subdirectories code and images respectively. You can redefine graphicspath and inputpath (see above in sourcecode) to change the code and image location.

This template was made for the course introduction to data science \cite{IDS2017}.


\section*{Exercise 1: Python example}

Three different lstsets are defined for Python, Matlab and R. The only difference between them are their keywordsets and are otherwise the same.

Always refer and add captions to your code. Example python code in listing \ref{lst:python_code} and matlab code in listing \ref{lst:matlab_code}.

\lstinputlisting[language=Python,label=lst:python_code,caption=Hello World Example]{code/example.py}

\lstinputlisting[language=Matlab,label=lst:matlab_code,caption=FizzBuzz example]{code/fizzbuzz.m}

\subsection{Association rule mining}

\section*{Exercise 3: R example}

Figure \ref{fig:rboxplot} was created by the script in listing \ref{lst:r_code}. The script was taken from tutorialspoint \cite{Rboxplot}.
\begin{figure}[h]
\begin{center}
\includegraphics[width=0.5\textwidth]{boxplot}
\caption{Boxplot example from mtcars data.}
\label{fig:rboxplot}
\end{center}
\end{figure}

\lstinputlisting[language=R,label=lst:r_code,caption=Boxplot example]{code/boxplot.R}

% citation section
\bibliographystyle{plain}
\bibliography{bibfile}

\end{document}