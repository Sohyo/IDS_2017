\documentclass[a4paper]{article}

%% Language and font encodings
\usepackage[english]{babel}
\usepackage[utf8x]{inputenc}
\usepackage[T1]{fontenc}
\usepackage{listings}

%% Sets page size and margins
\usepackage[a4paper,top=3cm,bottom=2cm,left=3cm,right=3cm,marginparwidth=1.75cm]{geometry}

%% Useful packages
\usepackage{amsmath}
\usepackage[colorinlistoftodos]{todonotes}
\usepackage[colorlinks=true, allcolors=blue]{hyperref}
\usepackage{listings}
\usepackage{url}
\usepackage{graphicx}
\graphicspath{ {./images/} }
% \DeclareGraphicsExtensions{.pdf,.jpg,.png}

%% defined colors
\definecolor{Blue}{rgb}{0,0,0.5}
\definecolor{Green}{rgb}{0,0.75,0.0}
\definecolor{LightGray}{rgb}{0.6,0.6,0.6}
\definecolor{DarkGray}{rgb}{0.3,0.3,0.3}

% \newcommand*\lstinputpath[1]{\lstset{inputpath=#1}}
% \lstinputpath{code/}

\lstset{language=R,
   basicstyle=\ttfamily\small,
   breaklines=true,
   keywordstyle=\bfseries\color{Blue},
   commentstyle=\itshape\color{LightGray},
   stringstyle=\color{Green},
   numbers=left,
   numberstyle=\tiny\color{DarkGray},
   stepnumber=1,
   numbersep=10pt,
   backgroundcolor=\color{white},
   tabsize=2,
   showspaces=false,
   showstringspaces=false,
   captionpos=b,
%    inputpath={code/},
   frame=tb
}

\lstset{language=Python,
   basicstyle=\ttfamily\small,
   breaklines=true,
   keywordstyle=\bfseries\color{Blue},
   commentstyle=\itshape\color{LightGray},
   stringstyle=\color{Green},
   numbers=left,
   numberstyle=\tiny\color{DarkGray},
   stepnumber=1,
   numbersep=10pt,
   backgroundcolor=\color{white},
   tabsize=2,
   showspaces=false,
   showstringspaces=false,
   captionpos=b,
%    inputpath={code/},
   frame=tb
}

\lstset{language=Matlab,
   basicstyle=\ttfamily\small,
   breaklines=true,
   keywordstyle=\bfseries\color{Blue},
   commentstyle=\itshape\color{LightGray},
   stringstyle=\color{Green},
   numbers=left,
   numberstyle=\tiny\color{DarkGray},
   stepnumber=1,
   numbersep=10pt,
   backgroundcolor=\color{white},
   tabsize=2,
   showspaces=false,
   showstringspaces=false,
   captionpos=b,
%    inputpath={code/},
   frame=tb
}

\title{IDS 2017 \\Assignment 1}
\author{
 Bogdan Petre (s3480941) \\ 
 Low Daniel (s3120155) \\
 Xu Teng Andrea (s3548120) 
 \\ \textbf{Group} 7}
\date{\today}


\begin{document}
\maketitle
\section{}
\subsection{Identify Data Types (10P)}

	\begin{itemize}
		\item Brightness as measured by a light meter: continuous because each measurement obtains a distinct score[1], quantitative (ratio) because this device could have an absolute zero (i.e., absence of light). 
		
		\item Brightness as measured by people’s judgments: if you use a Lickert scale to measure people’s judgment, then brightness would be discrete and qualitative (ordinal) becuase the measurements have a logical order but do not reflect numerical true values.  
		
		\item Time in terms of AM or PM: binary, qualitative (nominal if one considers there is not a logical order between AM and PM or ordinal if one views PM coming after AM).
		
		\item Coat check number (certain places offer you to leave your coat to someone who, in turn, gives
		you a number tag that you need to claim it back when you leave): discrete, qualitative (ordinal or perhaps nominal if the coats aren’t placed in the order of the integers).
	\end{itemize}

	\subsection{}
	\subsection{Think About Types (20P)}
	
	\begin{itemize}
		\item Title: discrete, qualitative (categorical)
		\item ReleaseDate: discrete, quantitative (interval)
		\item Popularity: continuous, quantitative (interval)
		\item Budget: continuous, quantitative (ratio)
		\item Revenue: continuous, quantitative (ratio)
		\item Genre: discrete, qualitative (categorical)
		\item imdbRating: discrete, qualitative (ordinal)
		\item imdbVotes: discrete, quantitative (ratio) 
		\item Director: discrete, qualitative (categorical) 
		\item Country: discrete, qualitative (categorical) 
		\item PG rating: discrete, qualitative (ordinal)
	\end{itemize}


% citation section
\newpage
\begin{thebibliography}{9}
	\bibitem{DiscoveringStatistics}
	Field, A. (2009). Discovering statistics using SPSS. Sage publications. 
	
\end{thebibliography}


\end{document}